\documentclass[a4paper,11pt]{article}
\usepackage[utf8]{inputenc}
\usepackage[czech]{babel}
\usepackage{graphicx}
\usepackage{float}
\usepackage{hhline}
\usepackage[unicode]{hyperref}
\usepackage{amsmath}
\pagenumbering{arabic}

\begin{titlepage}
\title{Hledání nejkratších cest v grafu}
\date{\today}
\author{Tomáš Duda a Artemij Pozdňakov \\ dudatom2@fit.cvut.cz a pozdnart@fit.cvut.cz}
\end{titlepage}

\begin{document}

% Uvodni stranka
\maketitle
\newpage

% Obsah
\tableofcontents
\newpage

% Definici problému
% Popis sekvenčního algoritmu a jeho implementace
\section{Řešený problém a základní implementace}
\subsection{Definice problému}
Naším úkolem v rámci semestrálního projektu v předmětu BI-EIA je implementace a optimalizace dvou algoritmů pro hledání nejkratších cest. Vstupem programu je tedy graf zadaný výčtem hran grafu a výstupem délka nejkratší cesty pro každou dvojici uzlů.
\par
Pro řešení problému jsme implementovali dva algoritmy, první je Dijk\-strův, druhý Floyd-Warshallův. Jejich stručný popis a informace o základní implementaci jsou obsahem ná\-sle\-dujících částí.

\subsection{Dijkstrův algoritmus}
Dijkstrův algoritmus funguje obdobně jako prohledávání do šířky, jenom místo obyčejné fronty používá prioritní frontu. Do té jsou před během algoritmu přesunuty všechny uzlu, počáteční z nulovou, zbylé s $\infty$ vzdáleností. Poté se až do vyprázdnění fronty vybírá nejbližší uzel a pro všechny jeho sousedy se vyzkouší, zda byla nalezena zkracující cesta (relaxace). 
\par
Běžná verze Dijkstrova algoritmu je určena pro hledání nejkratších cest z jednoho uzlu do všech ostatních, proto ho musíme v našem řešení volat $|V|$-krát, tedy z každého uzlu. Asymptotická složitost Dijkstrova algoritmu závisí na dvou faktorech. Jednak jde o vnitřní reprezentaci grafu (seznam uzlů a jejich sousedů nebo matice sousednosti) a dále o způsob implementace prioritní fronty, kterou algoritmus využívá.
\par
V rámci snahy o kompromis mezi rozumnou rychlostí a dostatečnou možností kód následně optimalizovat jsme použili kombinaci reprezentace grafu seznamem sousedů a prioritní fronty řešené pomocí binární haldy. Tato kombinace má při hledání nejkratších cest mezi všemi dvojicemi uzlů asymptotickou složitost $\mathcal{O}(|V|(|E|+|V|)\log{|V|})$, kde V je množina uzlů a E je množina hran.

\subsection{Floyd-Warshallův algoritmus}
Floyd-Warshallův algoritmus hledá nejkratší cesty metodou postupné konstrukce. Skládá se ze tří do sebe vnořených cyklů, které iterují přes všechny uzly grafu. Vnější cyklus určuje prostředníka, přes které se algoritmus právě snaží nalézt zlepšující cestu, dva vnitřní cykly poté určují dvojici koncových uzlů.
Jako vnitřní reprezentace grafu je použita matice sousednosti, která je postupem algoritmu přepsaná na výslednou distanční matici. Asymptotická složitost Floyd-Warshallova algoritmu je $\mathcal{O}(|V|^3)$

\subsection{Porovnání algoritmů}
Pro řídké grafy ($|E|\sim|V|$) má Dijkstrův algoritmus (s použitím binární haldy) nižší teoretickou složitost než Floyd-Warshallův. Dá se však očekávat, že při reálném použití u grafů, které jsme schopni v rozumném čase na poskytnutém HW upočítat (řádově tisíce uzlů), bude Floyd-Warshallův algoritmus rychlejší, jelikož potřebuje vykonat v nejvnitřnějším cyklu mnohem méně operací než Dijkstrův.

\subsection{Popis souborů}
\begin{itemize}
 \item \textbf{main.cpp} obsahuje zpracování argumentů z příkazové řádky a volání algoritmů.
 \item \textbf{mygraph.cpp} obsahuje třídu MyGraph, která jednak slouží pro vni\-třní reprezentaci grafu (podporuje jak matici sousednosti tak i reprezentaci pomocí seznamů uzlů a jejich sousedů). Dále realizuje zpracování vstupního grafu ze souboru.
 \item \textbf{node.cpp} a \textbf{edge.cpp} obsahují pomocné třídy pro uložení uzlu nebo hrany v grafu.
 \item \textbf{floydwarshall.cpp} implementuje Floyd-Warshallův algoritmus.
 \item \textbf{dijkstra.cpp} implementuje Dijkstrův algoritmus.
 \item \textbf{exception.cpp} definuje výjimku, která je použita při spuštění programu s chybnými argumenty.
\end{itemize}




% Popis případných úprav algoritmu a jeho implementace, včetně volby datových struktur
% Tabulkově a případně graficky zpracované naměřené hodnoty časové složitosti měřených instancí běhu programu s popisem instancí dat, přepočet výkonnosti programu na MIPS nebo MFlops.
% Analýza a hodnocení vlastností dané implementace programu.
\section{Optimalizovaná verze sekvenčního algoritmu}
Následující kapitola popisuje optimalizace provedené v sekvenčních verzích implementace a následné výsledky meření různých verzí.

\subsection{Dijkstrův algoritmus}
Jelikož nebylo Dijkstrův algoritmus možné optimalizovat klasickými metodami (vysoká datová provázanost, nemožnost rozbalit vnitřní cyklus kvůli komplikované datové struktuře), pokusili jsme se řešení optimalizovat dvěma jinými způsoby. 
\par

Prvním byla výměna původně použité prioritní fronty z STL za vlastní implementaci, která navíc podporuje operaci decreaseKey a tudíž není po\-třeba u každého uzlu vyňatého z fronty testovat, zda je jeho hodnota klíče aktuální (zkrátka není nutné používat reinserting).
\par
Druhý pokus o optimalizaci proběhnul pomocí použití různých přepínačů při kompilaci v gcc.
\begin{itemize}
 \item \textbf{-O3} - zapnutí plných optimalizací cílového kódu.
 \item \textbf{-march=opteron} - využití všech instrukcí na cílovém procesoru.
 \item \textbf{-mpc32} - zaokrouhlení FP výpočtů.
 \item \textbf{-msseregparm} - použití SSE registrů pro předání parametrů funkcí. Chtěli jsme použít i \textbf{mregparm=3}, což zablokovalo g++ (-mregparm is ignored in 64-bit mode). 
 \item \textbf{-mffast-math} - zrychlené vyhodnocení matematických výrazů.
 \item Naopak jsme nepoužili vektorové instrukce SSE. Potvrdila se do\-mněn\-ka, že v Dijkstrově algoritmu jsou zbytečné a jejich použití program zpomalí.
\end{itemize}
TODO.

\subsection{Floyd-Warshallův algoritmus}
Loop-tiling, loop unrolling, vektorizace? TODO.

\subsection{Popis testovacích instancí}
Výběr testovacích instancí pro testování a porovnání obou implementovaných algoritmů byl poměrně komplikovaný. Dijkstrův algoritmus je narozdíl od Floyd-Warhsallova citlivý na vstupní data, tudíž nylo nutné, aby se jednotlivé testovací grafy lišily nejenom v počtu uzlů, ale i v počtu hran. 
\par
Floyd-Warshallův algoritmus se navíc ukázal být o hodně výkonnějším, tudíž zatímco neoptimalizovaná implementace Dijkstrova algoritmu na nej\-větší instanci za 60 minut na testovacím serveru nedoběhla, Floyd-Warshall ji upočítal za 10 minut.
\par
Vybrali jsme tedy grafy o čtyřech různých velikostech, co do počtu uzlů (800, 1600, 2400 s 3200) a u každého z nich tři různé instance lišící se poštem hran. První z trojice je vždy řídký graf (obsahuje přibližně desetinu hran, co graf úplný), druhý obsahuje přibližně polovinu hran úplného grafu a třetí je hustý, obsahuje 90 \% hran úplného grafu.
\begin{table}[H]
  \begin{center}
      \begin{tabular}{|r|r|r|}
      \hline
      Název souboru 	& Počet uzlů  	& Počet hran  		\\ \hline
      graf800\_80  	& 800    	& 31521             	\\ \hline
      graf800\_400     	& 800    	& 160035        	\\ \hline
      graf800\_720  	& 800    	& 287590             	\\ \hline
      graf1600\_160    	& 1600    	& 128022          	\\ \hline
      graf1600\_800  	& 1600    	& 639663             	\\ \hline
      graf1600\_1400   	& 1600    	& 1151713          	\\ \hline
      graf2400\_240  	& 2400    	& 289420             	\\ \hline
      graf2400\_1200   	& 2400    	& 1439793          	\\ \hline
      graf2400\_2160  	& 2400    	& 2591136             	\\ \hline
      graf3200\_320    	& 3200    	& 513016          	\\ \hline
      graf3200\_1600  	& 3200    	& 2560196             	\\ \hline
      graf3200\_2880   	& 3200    	& 4608704          	\\ \hline
      \end{tabular}
  \caption{Vlastnosti grafů, na kterých byly testovány sekvenční implementace algoritmů.}
  \label{tab:instance}
  \end{center}
\end{table}


\subsection{Měření a porovnání výkonosti různých sekvenčních verzí}
V následující části jsou uvedeny výsledky meření jednotlivých typů implementací. Údaje jsou uváděny jednak v sekundách (reálná doba běhu na serveru STAR) a v MFPLOPS. Zatímco i Floyd-Warshallova algoritmu můžeme počítat s~$2|V|^3$ operací v plovoucí čárce na jeden běh, u Dijkstrova algoritmu je situace o trochu komplikovanější, protože počet operací v FP závisí na podobě vstupního grafu a odhady pomocí složitosti nebyly příliš přesné. Změřili jsme tedy počet FP operací pro jednotlivé testovací instance.
\begin{table}[H]
  \begin{center}
      \begin{tabular}{|r|r|}
      \hline
      Název souboru 	& FP operací  	\\ \hline
      graf800\_80  	& 115,283 M            		\\ \hline
      graf800\_400     	& 524,098 M    		     	\\ \hline
      graf800\_720  	& 932,637 M    	            	\\ \hline
      graf1600\_160    	& 879,441 M    	        	\\ \hline
      graf1600\_800  	& 4,139 G    	           	\\ \hline
      graf1600\_1400   	& 7,414 G    	         	\\ \hline
      graf2400\_240  	& 2,912 G    	          	\\ \hline
      graf2400\_1200   	& 13,929 G    	         	\\ \hline
      graf2400\_2160  	& 24,956 G    	            	\\ \hline
      graf3200\_320    	& 6,800 G             		\\ \hline
      graf3200\_1600  	& 32,963 G    	            	\\ \hline
      graf3200\_2880   	& 59,136 G    	         	\\ \hline
      \end{tabular}
  \caption{Počet FP operací potřebných k běhu Dijkstrova algoritmu pro jednotlivé instance.}
  \label{tab:fp_operace}
  \end{center}
\end{table}

\subsubsection{Měření Dijkstrova algoritmu}
V následující tabulce jsou naměřené hodnoty pro Dijkstrův algoritmus. Mě\-ře\-na byla jednak verze, která využívá prioritní frontu z STL, poté verze používající prioritní frontu s podporou operace decreaseKey, třetí verze byla kompilovaná s pře\-pí\-načem \textbf{-O3} a nakonec verze, ve které byly použity další pře\-pí\-nače vypsané v sekci Optimalizovaná verze sekvenčních algoritmů.
\par
Kromě dramatického nárustu výkonu při použití optimalizací kom\-pi\-látoru jde z měření vypozorovat, že Dijkstrův algoritmus je nejsilnější na menších grafech s nízkým počtem hran.
\begin{table}[H]
  \begin{center}
      \begin{tabular}{|r|r|r|r|r|}
      \hline
      Instance  	& STL	  & BH    & -O3	 & g++ opt  \\ \hline
      graf800\_80  	& 11,9  & 20,8	  & 50,8 &          \\ \hline
      graf800\_400     	& 15,5  & 16,7 	  & 25,5 &	    \\ \hline
      graf800\_720  	& 13,0  & 15,9 	  & 28,9 &	    \\ \hline
      graf1600\_160    	& 12,3  & 16,9 	  & 28,4 &	    \\ \hline
      graf1600\_800  	& 14,4  & 15,6 	  & 26,6 &	    \\ \hline
      graf1600\_1400   	& 14,8  & 16,4 	  & 25,4 &	    \\ \hline
      graf2400\_240  	& 11,8  & 15,3	  & 25,1 &          \\ \hline
      graf2400\_1200   	& 14,2  & 15,4	  & 24,8 &          \\ \hline
      graf2400\_2160  	& 14,8  & 12,6	  & 25,1 &          \\ \hline
      graf3200\_320    	& 12,3  & 12,7	  & 23,7 &          \\ \hline
      graf3200\_1600  	& 14,3  & 14,9	  & 25,0 &          \\ \hline
      graf3200\_2880   	& x    	& x 	  & 25,1 &          \\ \hline
      \end{tabular}
  \caption{Naměřené výsledky pro Dijkstrův algoritmus. Hodnoty jsou v~MFLOPS. Políčka, ve kterých je uvedeno x značí, že pro ně algoritmus nedokázal doběhnout ve stanoveném limitu 60 minut.}
  \label{tab:dijkstra1}
  \end{center}
\end{table}


\subsubsection{Měření Floyd-Warshallova algoritmu}
TODO popis výsledků a efektivity optimalizací.
\begin{table}[H]
  \begin{center}
      \begin{tabular}{|r|r|r|r|r|}
      \hline
      Instance  	& Základní  & -O3     & ???	 & ???  \\ \hline
      graf800\_80  	& 129,0     & 975,2   &  &          \\ \hline
      graf800\_400     	& 130,4     & 890,4   &  &	    \\ \hline
      graf800\_720  	& 129,3     & 1402,7  &  &	    \\ \hline
      graf1600\_160    	& 125,2     & 686,7   &  &	    \\ \hline
      graf1600\_800  	& 125,3     & 688,4   &  &	    \\ \hline
      graf1600\_1400   	& 125,2     & 680,9   &  &	    \\ \hline
      graf2400\_240  	& 126,0     & 670,7   &  &          \\ \hline
      graf2400\_1200   	& 126,1     & 671,0   &  &          \\ \hline
      graf2400\_2160  	& 126,4     & 670,0   &  &          \\ \hline
      graf3200\_320    	& 125,8     & 665,6   &  &          \\ \hline
      graf3200\_1600  	& 126,0     & 663,6   &  &          \\ \hline
      graf3200\_2880   	& 125,9     & 662,9   &  &          \\ \hline
      \end{tabular}
  \caption{Naměřené výsledky pro Floyd-Warshallův algoritmus. Hodnoty jsou v~MFLOPS.}
  \label{tab:fw1}
  \end{center}
\end{table}



% Popis případných úprav algoritmu a jeho implementace, včetně volby datových struktur
% Tabulkově a případně graficky zpracované naměřené hodnoty časové složitosti měřených instancí běhu programu s popisem instancí dat, přepočet výkonnosti programu na MIPS nebo MFlops.
% Analýza a hodnocení vlastností dané implementace programu.
\section{Vícevláknová implementace}

% Závěr
\section{Závěr}


\newpage
\begin{thebibliography}{1}
  \bibitem[1]{Kolar} KOLÁŘ, Josef.
    \emph{Teoretická informatika}.
    Česká informatická společnost, Praha, 2004. 205s.
\end{thebibliography}

\end{document}

